%
% Ejemplo detallado de ejecución de una función recursiva: el factorial
% (compilar con xelatex -8bit -shell-escape)
%
% Rubén Rubio (Universidad Complutense de Madrid) 2023
%

\documentclass[aspectratio=169]{beamer}

\beamertemplatenavigationsymbolsempty

\usepackage{fontspec}
\setmainfont{Source Sans 3}
\setsansfont{Source Sans 3}
\setmonofont{Inconsolata}
\usepackage{unicode-math}
\setmathfont{Fira Math}

\usepackage{tikz}
\usepackage{minted}
\usepackage{colortbl}

\setminted{tabsize=3, highlightcolor=yellow!60, linenos}

\begin{document}

\begin{frame}[fragile]{Ejecución detallada del factorial recursivo}
% Selecciona la línea que se resalta del código
\only<1,12>    {\setminted{highlightlines={7}}}
\only<2,4,6>   {\setminted{highlightlines={2}}}
\only<3,5,8-11>{\setminted{highlightlines={5}}}
\only<7>       {\setminted{highlightlines={3}}}

\begin{minted}[]{python}
def factorial(n):
	if n == 0:
		return 1
	else:
		return n * factorial(n - 1)

factorial(2)
\end{minted}

\begin{tikzpicture}[overlay]
\fill[white] (7, 0) rectangle (14.2, 4);
\node<2,3,10,11>[font=\small, fill=green!40, rounded corners=2pt] at (4, 3.6) {$n = 2$};
\node<4,5,8,9>[font=\small, fill=green!40, rounded corners=2pt] at (4, 3.6) {$n = 1$};
\node<6,7>[font=\small, fill=green!40, rounded corners=2pt] at (4, 3.6) {$n = 0$};
\node<9,11>[font=\small, fill=green!40, rounded corners=2pt, text width=3cm, align=center] at (4.8, 1.63) {$1$};
\node<12>[font=\small, fill=green!40, rounded corners=2pt] at (3, .675) {$= 2$};
\node[anchor=north west, fill=white, text width=4cm] at (8.5, 4) {
\textbf{Pila de llamadas}
\vspace{.6ex}
\only<1,11>{\vspace{-2.2pt}}

% Pila de llamadas
\begin{tabular}{|l|p{2cm}|p{1.5cm}|} \hline
\# & Variables & Retorno%
\only<2-9>{\\ \hline 1 & $n = 2$ & línea 7}%
\only<10,11>{\\ \hline 1 & $n = 2$ & \cellcolor{yellow!60} línea 7}%
\only<4-7>{\\ \hline 2 & $n = 1$ & línea 5}%
\only<8,9>{\\ \hline 2 & $n = 1$ & \cellcolor{yellow!60} línea 5}%
\only<6>{\\ \hline 3 & $n = 0$ & línea 5}%
\only<7>{\\ \hline 3 & $n = 0$ & \cellcolor{yellow!60} línea 5}%
\\ \hline%
\end{tabular}};

% Mensajes explicativos de cada pantalla
\node[color=teal, text width=.9\linewidth, align=center] at (.5\linewidth, -1.5) {\itshape%
\only<1>{Se ejecuta la llamada \texttt{factorial(2)}}%
\only<2>{Se reserva espacio en la pila de llamadas para la nueva variable n y se guarda el punto (línea 7) al que volver cuando acabe la llamada}%
\only<3>{Se necesita el valor de \texttt{factorial(n-1)}, así que dejamos la evaluación de la línea 5 pendiente y llamamos de nuevo a \texttt{factorial} con $n - 1 = 1$}%
\only<4>{En la pila queda registrado que \texttt{factorial(2)} está pendiente y tendremos que volver a él}%
\only<5>{Se repite ahora la misma historia, necesitamos obtener el \texttt{factorial} para $n \!-\! 1 \!\!=\!\! 0$ para calcular la expresión de la línea 5}%
\only<6>{Hemos llegado por fin al caso base, y \\ se ejecuta ahora la primera rama del condicional}%
\only<7>{Como hemos llegado a un \texttt{return}, termina esta llamada y  \\ volvemos al punto de retorno registrado en la pila devolviendo 1}%
\only<8,9>{Ya tenemos el valor de \texttt{factorial(0)} y podemos resolver el producto, \\ \texttt{factorial(1) = 1} y volvemos a la siguiente llamada en la pila}%
\only<10,11>{Se repite el mismo proceso, termina \texttt{factorial(2)} y volvemos a donde se hizo la llamada inicial}%
\only<12>{Finalmente hemos obtenido que el resultado es 2}%
};
\end{tikzpicture}

% Avanza el número de diapositiva
% (como todo es un overlay sobre la misma saldría siempre 1)
\only<2->{\stepcounter{framenumber}}
\end{frame}

\end{document}
